% !TeX program = pdflatex
\documentclass[aspectratio=169]{beamer}

% ---------- Lingua & font ----------
\usepackage[italian]{babel}
\usepackage[T1]{fontenc}
\usepackage[utf8]{inputenc}

% ---------- Pacchetti utili ----------
\usepackage{siunitx}
\usepackage{physics}
\usepackage{graphicx}
\usepackage{booktabs}
\usepackage{tikz}
\usepackage{pgfplots}
\pgfplotsset{compat=1.18}
\sisetup{detect-all, range-phrase=--, separate-uncertainty=true}
\usepackage{adjustbox}

% ---------- Palette "LED" ----------
\definecolor{ledRed}{HTML}{FF4D4F}
\definecolor{ledOrange}{HTML}{FF9100}
\definecolor{ledYellow}{HTML}{FFEA00}
\definecolor{ledGreen}{HTML}{00C853}
\definecolor{ledBlue}{HTML}{2979FF}
\definecolor{ledViolet}{HTML}{7C4DFF}
\definecolor{ledDark}{HTML}{0D1321}
\definecolor{ledGrey}{HTML}{ECEFF1}

% ---------- Tema ----------
\usetheme{Madrid}
\usecolortheme{seahorse}
\setbeamercolor{structure}{fg=ledBlue}
\setbeamercolor{frametitle}{bg=ledBlue!15!white, fg=ledDark}
\setbeamercolor{title}{fg=ledDark}
\setbeamercolor{block title}{bg=ledBlue!15!white,fg=ledDark}
\setbeamercolor{block body}{bg=ledGrey!40!white,fg=black}
\setbeamertemplate{navigation symbols}{}
\setbeamertemplate{footline}{%
	\leavevmode%
	\hbox{%
		\begin{beamercolorbox}[wd=.8\paperwidth,ht=2.5ex,dp=1ex,left]{author in head/foot}%
			\hspace{1ex}\insertshorttitle{} \textcolor{gray}{|} \insertshortauthor{} \textcolor{gray}{|} \insertshortinstitute{}
		\end{beamercolorbox}%
		\begin{beamercolorbox}[wd=.2\paperwidth,ht=2.5ex,dp=1ex,right]{date in head/foot}%
			\insertframenumber{} / \inserttotalframenumber\hspace{1ex}
	\end{beamercolorbox}}%
}

% ---------- Bullet tondi "LED" ----------
\setbeamertemplate{itemize item}{\tikz\fill[ledBlue] (0,0) circle (2pt);}
\setbeamertemplate{itemize subitem}{\tikz\fill[ledGreen] (0,0) circle (1.6pt);}
\setbeamertemplate{itemize subsubitem}{\tikz\fill[ledViolet] (0,0) circle (1.4pt);}

% ---------- Macro segnaposto ----------
\newcommand{\placeholder}[2][0.9\linewidth]{%
	\begin{center}
		\fbox{\parbox{#1}{\centering\small\textit{#2}}}
	\end{center}
}

% ---------- Comodità simboli ----------
\newcommand{\Vsoglia}{V_{\text{soglia}}}
\newcommand{\Isoglia}{I_{\text{soglia}}}
\newcommand{\Eg}{E_{\mathrm{g}}}
\newcommand{\VT}{V_{\!T}}
\newcommand{\Ipd}{I_{\mathrm{PD}}}
\newcommand{\Iled}{I_{\mathrm{LED}}}

\setbeamertemplate{enumerate items}[default]

% ---------- Titoli ----------

%Visto che la presentazione fa riferimento all'esperienza globlale di optoelettronico, forse avrebbe più senso fare il titolo accordingly

%Esempio: {Esperienza di optoelettronica: LED e fotodiodi & (---titolo seconda parte esperienza---)}

\title[Optoelettronica]{Optoelettronica: I–V dei LED, stima di $h$, fotodiodo e transimpedenza, verifica della legge di Lambert-Beer.}
\subtitle{Corso di Tecnologie Digitali — Università di Pisa, A.A.\ 2025/2026}
\author[Di Nino \& Malucchi (T10)]{Alessia Di Nino \and Marco Malucchi\\\small Tavolo T10}
\institute[UniPI]{Dipartimento di Fisica \textit{``E. Fermi''}}
\date{\today}
% \titlegraphic{\includegraphics[height=1cm]{unipi_logo.pdf}} % (opzionale) inserire logo

% ---------- Titolo con "barre LED" ----------
\addtobeamertemplate{title page}{%
	\vspace*{-1ex}
	\begin{center}
		\begin{tikzpicture}[x=0.7cm]
			\foreach \i/\c in {0/ledRed,1/ledOrange,2/ledYellow,3/ledGreen,4/ledBlue,5/ledViolet}{
				\fill[\c] (\i,0) circle (0.12);
				\fill[\c] (\i,0.35) circle (0.12);
				\fill[\c] (\i,0.7) circle (0.12);
			}
		\end{tikzpicture}
	\end{center}
}{}

% =========================================================
\begin{document}
	
	\begin{frame}
		\titlepage
	\end{frame}
	
	%Anche in questo caso, visto che dobbiamo trattare parte 1 e 2, io metterei una slide per la parte 1, dove ci mettiamo li obiettivi di quella parte, poi trattatiamo tutta la parte 1 e finita quella facciamo la stessa cosa per la parte 2.
	
	% ---------- Outline parte 1 ----------
	\begin{frame}{Parte 1: Obiettivi e outline}
		\begin{block}{Passaggi principali}
			\begin{itemize}
				\item Specifiche del LED: sue caratteristiche di funzionamento e curva caratteristica I-V.
				\item Stima della costante di Planck $h$.
				\item Fotodiodo OSD15-5T: test $V_{\!OC}$ e $I_{\!SC}$, poi misura con transimpedenza.
				\item Verifica della linearità $\Ipd \propto \Iled$ e discussione limiti sperimentali.
			\end{itemize}
		\end{block}
	\end{frame}
	
	% ---------- Teoria minima LED ----------
	\begin{frame}{Concetti teorici di base.}
		\begin{itemize}
			
			\item LED (light emitting diode): diodo a giunzione pn, la luce viene emessa dai processi di ricombinazione delle cariche minoritarie che attraversano la giunzione quando è polarizzata direttamente.
			\pause
			\item Energia del fotone: $E=\dfrac{hc}{\lambda}$; visibile $\lambda\sim 400$--$700\,$nm $\Rightarrow E\sim 3$--$1.7\,$eV, assumo che l'energia dei fotoni emessi sia pari all'energia del band gap $E_g$.
			\pause
			\item Equazione di Shockley: $I=I_S\!\left(e^{V/\VT}-1\right)$, con $\VT \simeq \SI{26}{mV}$ a T ambiente, mi aspetto che la curva caratteristica del LED abbia questo andamento.
		\end{itemize}
	\end{frame}
	
	% ---------- Specifiche LED ----------
	\begin{frame}{Specifiche LED}
		\begin{columns}[T]
			\column{0.52\linewidth}
			
			Obiettivo: determinare la caratteristica del diodo, vedi schema del circuito:
			
			\begin{itemize}
				\item Potenziale ai capi del LED:
				\[
				V_{\text{LED}}=V_{\text{Ch2}}
				\]
				\item Corrente ai capi del LED:
				\[
				\Iled=\dfrac{V_{\text{Ch1}}-V_{\text{Ch2}}}{R_1}
				\]
				\item Sweep in tensione (0–3/4 V), $\sim$100 punti; grafico semi-log di $I$ vs $V$.
			\end{itemize}
			\column{0.44\linewidth}
			\begin{figure}
				\centering
				\includegraphics[width=\linewidth]{circuito_task2_es_06_LED.png}
				\caption{Schema del circuito}
				\label{fig:scema_circuito_task2}
			\end{figure}
		\end{columns}
	\end{frame}
	
	%---Specifiche LED e curva caratteristica misurata---
	
	\begin{frame}{Specifiche LED}
		
		Specifiche LED rosso dal datasheet:
		
		\begin{itemize}
			\item Corrente di funzionamento: $I = 20$ mA
			\item Voltaggio di funzionamento: $VS = 2 - 2.2$ V
			\item Intensità luminosa (in millicandles "mcd"): $I_{lum} = 600 - 800$ mcd
		\end{itemize}
		\pause
		Infine i valori di ddp e corrente per i quali il LED inizia a brillare:
		
		\begin{itemize}
			\item $V_{soglia} = 1.5$ V.
			\item $I_{soglia} = 20 \mu$ A.
		\end{itemize}
		
	\end{frame}
	
	\begin{frame}[t]
		\frametitle{Curva caratteristica}
		
		\begin{figure}
			\centering
			% Usa il codice che avevi (con height=0.75\textheight)
			\includegraphics[
			width=0.9\textwidth,
			height=0.75\textheight,
			keepaspectratio
			]{characteristic_curve_LED_rosso.png}
			
			\vspace{-0.5cm} % <-- AGGIUNGI QUESTO. Tira su quello che segue
			%     (prova con -0.5cm, -1cm, -1.5cm...)
			
			\caption{Curva caratteristica LED rosso in scala semilog.}
			\label{fig:curva_caratteristica_rosso}
		\end{figure}
	\end{frame}
	
	%---STIMA COSTANTE DI PLANCK---
	
	\begin{frame}{Stima della costante di Planck}
		
		\begin{columns}[T]
			\column{0.52\linewidth}
			
			Facendo le seguenti assunzioni:
			\begin{itemize}
				\item $I_s \propto {n_i}^2 \propto e^{\frac{-E_g}{k_B T}}$, con $E_g=\frac{hc}{\lambda}$
				\item $V \gg V_T$
			\end{itemize}
			\pause
			Trovo:
			
			\[
			I \propto e^{\frac{V-V_g}{V_T}}
			\]
			
			con $V_g \equiv \frac{E_g}{e}$
			\pause
			
			Allora arbitrariamente fissata $I_{soglia}$ e posto $V_{soglia} \equiv V(I_{soglia})$:
			
			\[
			V_{soglia}-V_g = V(I_{soglia}) - \frac{hc}{e\lambda}
			\]
			
			è una \textbf{costante} per ogni LED
			
			\pause
			
			\column{0.44\linewidth}
			
			Stima di $h$:
			
			\begin{enumerate}
				\item Trovare $V_{soglia}$ e $\lambda$ per ogni LED
				\pause
				\item Fittare dati con funzione:
				\[
				\frac{1}{\lambda} = \frac{e}{hc} V(I_{soglia}) + b
				\]
				\pause
				\item Stimare, dal coefficiente angolare modello lineare, $h$
			\end{enumerate}
		\end{columns}
	\end{frame}
	
	
	% ---------- Spettri ----------
	\begin{frame}{Stima della costante di Planck: lunghezza d'onda LED}
		
		Per determinare $\frac{1}{\lambda}$, uso gli spettri dei LED utilizzati messi a disposizione.
		\pause
		Li manipolo nel seguente modo:
		
		\begin{enumerate}
			\item Rimuovo l'offset.
			\pause
			\item Tratto lo spettro come distribuzione, quindi ricavo media e deviazione standard come:
			\[
			\langle \lambda \rangle = \begin{matrix} \sum_{i} \lambda_i \frac{I(\lambda_i)}{I_{TOT}} \end{matrix} \; ; \; \sigma_{\lambda} = HWHM
			\]
			\pause
			quindi $\langle \lambda \rangle$ come media pesata con intensità luminosa relativa al totale e dev. std. come larghezza a metà altezza del picco principale
		\end{enumerate}
	\end{frame}
	
	\begin{frame}{Stima della costante di Planck: esempio spettro LED e risultati}
		\begin{columns}[T]
			\column{0.52\linewidth}
			
			Per contestualizzare i conti sopra:
			\begin{figure}
				\centering
				\includegraphics[width=\linewidth]{esempio_spettro_4.png}
				\caption{Esempio di spettro dei LED utilizzati.}
				\label{fig:esempio_spettro}
			\end{figure}
			
			\column{0.44\linewidth}
			\pause
			Di seguito i risultati ottenuti per $\langle \lambda \rangle$ e $\sigma_{\lambda}$
			
			\begin{table}[h!]
				\centering
				\scriptsize
				\begin{tabular}{c c c}
					%\begin{tabular}{|c|c|c|}
					\hline
					\textbf{LED} & \textbf{Media [nm]} & \textbf{Dev. std. [nm]} \\
					\hline
					Rosso      & 625 & 7 \\
					Blu        & 469 & 7 \\
					Giallo     & 588 & 7 \\
					Verde      & 532 & 1 \\
					Arancione  & 604 & 8 \\
					Viola      & 483 & 6 \\
					\hline
				\end{tabular}
			\end{table}
			
		\end{columns}
	\end{frame}
	
	
	
	% ---------- Vsoglia ----------
	\begin{frame}{Stima della costante di Planck: ottenimento delle $\Vsoglia$}
		
		Procedimento per ottenere i valori di $\Vsoglia$ per ogni LED:
		\pause
		
		\begin{enumerate}
			\item Per tutti i LED si è scelto il seguente valore di $\Isoglia$: $\Isoglia = 100 \mu A$
			\pause
			\item Per ogni LED è stata acquisita la sua curva caratteristica e ricavato $\Vsoglia$
			\pause
			\item I valori di $\Vsoglia$ sono stati presi selezionando il punto più vicino al valore di $\Isoglia$ per ogni curva caratteristica
			\pause
		\end{enumerate}
		
		Di seguito i valori ottenuti:
		
		\begin{table}[h!]
			\centering
			\scriptsize
			\begin{tabular}{c c c}
				\hline
				\textbf{LED} & \textbf{V($I_{\mathrm{soglia}}$) [V]} & \textbf{Incertezza [V]} \\
				\hline
				Rosso      & 1.710 & 0.005 \\
				Arancione  & 1.769 & 0.006 \\
				Giallo     & 1.798 & 0.004 \\
				Verde      & 2.210 & 0.005 \\
				Blu        & 2.501 & 0.005 \\
				Viola      & 2.885 & 0.006 \\
				\hline
			\end{tabular}
		\end{table}
		
	\end{frame}
	
	% ---------- Risultati fit per Planck constant ----------
	
	\begin{frame}{Stima della costante di Planck: fit e relativi risultati}
		
		Risultati del fit:
		
		\begin{table}[h!]
			\centering
			\begin{tabular}{c c c}
				\hline
				\textbf{Parametro} & \textbf{Valore} & \textbf{Valore atteso} \\
				\hline
				${\chi}^2_{\mathrm{rid}}$ & $10.2 \pm 0.7$ & $1$ \\
				$a$ & $(4.3 \pm 0.2)\cdot 10^{5}~\mathrm{C\,J^{-1}m^{-1}}$ & $806554~\mathrm{C\,J^{-1}m^{-1}}$ \\
				$b$ & $(9.0 \pm 0.4)\cdot 10^{5}~\mathrm{m^{-1}}$ & difficile da stimare \\
				\hline
			\end{tabular}
		\end{table}
		
		Dove $a \equiv coeff.ang$ e $b \equiv offset$.
		
		Costante di planck:
		\[
		h = \frac{e}{ac} = (1.24 \pm 0.06) \cdot 10^{-33} J \cdot s
		\]
		a circa 9 barre d'errore dal valore effettivo: $h \simeq 6.6 \cdot 10^{-34} J \cdot s$
		
	\end{frame}
	
	\begin{frame}{Stima della costante di Planck: fit e relativi risultati}
		
		\begin{figure}
			\centering
			% Usa il codice che avevi (con height=0.75\textheight)
			\includegraphics[
			width=0.9\textwidth,
			height=0.75\textheight,
			keepaspectratio
			]{plot_fit_Vsoglia_vs_1suLambda1.png}
			
			\vspace{-0.5cm} % <-- AGGIUNGI QUESTO. Tira su quello che segue
			%     (prova con -0.5cm, -1cm, -1.5cm...)
			
			\caption{Grafici di fit e residui.}
			\label{fig:fit_graph_planck}
		\end{figure}
		
	\end{frame}
	
	% ---------- Fotodiodo: principio ----------
	\begin{frame}{Fotodiodo: principio e modelli}
		
		\begin{itemize}
			\item Il fotodiodo si comporta come un diodo la cui caratteristica è traslata di una fotocorrente:
			\pause
			\[
			I = I_S\!\left(e^{V/\VT}-1\right) - I_L
			\]
			
			\pause
			\item In luce monocromatica, la corrente fotogenerata è proporzionale alla potenza ottica incidente:
			\pause
			\[
			I_L = \eta(\nu)\,\frac{e}{h\nu}\,W \quad \Rightarrow \quad R = \frac{\eta e}{h\nu}
			\]
			
			\pause
			\item Si distinguono due regimi operativi fondamentali:
			\pause
			\[
			\text{Regime fotovoltaico: } V_{\!OC}, \qquad 
			\text{Regime fotoconduttivo: } I_{\!SC}.
			\]
		\end{itemize}
		
	\end{frame}
	
	% ---------- Transimpedenza ----------
	\begin{frame}{Amplificatore a transimpedenza}
		
		\begin{columns}[T]
			% ====== Colonna sinistra: testo + schema ======
			\column{0.55\linewidth}
			\begin{figure}
				\centering
				\includegraphics[width=0.4\linewidth]{amplificatore_a_transimpedenza_schema.png}
				\caption{Schema dell’amplificatore a transimpedenza}
			\end{figure}
			
			\pause
			
			\begin{itemize}
				\item Output: $V_{\text{out}} = -R_2\,I_{\mathrm{PD}}$ \quad 
				\item MCP601/602 alimentati a $\pm\SI{3}{V}$; aggiungere $C \parallel R_2$ per stabilità.
				\item Tipico: $R_2 = \SI{10}{k\ohm}$, $C \sim \SI{1}{nF}$.
			\end{itemize}
			% ====== Colonna destra: foto circuito ======
			\pause
			\column{0.42\linewidth}
			\begin{figure}
				\centering
				\includegraphics[width=\linewidth]{amplificatore_a_transimpedenza_realizzato.jpg}
				\caption{Circuito realizzato su breadboard}
			\end{figure}
			
		\end{columns}
	\end{frame}
	
	
	\begin{frame}{Segnali acquisiti e analisi del rumore (1/1)}
		\textbf{Configurazioni testate:}
		
		\begin{columns}
			% Configurazione 1: Fotodiodo coperto
			\begin{column}{0.48\textwidth}
				\begin{enumerate}
					\item \textbf{Fotodiodo coperto (buio):}
					\begin{figure}
						\centering
						\includegraphics[width=0.95\linewidth]{coperto_task7.png}
						\caption{Segnale al buio (Rumore di fondo)}
					\end{figure}
				\end{enumerate}
			\end{column}
			
			% Configurazione 2: Luce ambiente
			\begin{column}{0.48\textwidth}
				\begin{enumerate}
					\setcounter{enumi}{1}
					\item \textbf{Luce ambiente della stanza:}
					\begin{figure}
						\centering
						\includegraphics[width=0.95\linewidth]{stanza_task7.png}
						\caption{Segnale con luce ambiente}
					\end{figure}
				\end{enumerate}
			\end{column}
		\end{columns}
		
		In buio: segnale stabile, $\sigma_V \approx 10\text{ mV}$ (rumore elettronico).
	\end{frame}
	
	\begin{frame}{Segnali acquisiti e analisi del rumore (2/2)}
		
		\begin{columns}
			% Configurazione 3: Lampada del treno
			\begin{column}{0.48\textwidth}
				\begin{enumerate}
					\setcounter{enumi}{2} % Inizia la numerazione da 3
					\item \textbf{Lampada:}
					\begin{figure}
						\centering
						\includegraphics[width=0.95\linewidth]{lampada_treno_task7.png}
						\caption{Segnale lampada }
					\end{figure}
				\end{enumerate}
			\end{column}
			
			% Configurazione 4: Flash del telefono
			\begin{column}{0.48\textwidth}
				\begin{enumerate}
					\setcounter{enumi}{3} % Inizia la numerazione da 4
					\item \textbf{Flash del telefono:}
					\begin{figure}
						\centering
						\includegraphics[width=0.95\linewidth]{flash_task7.png}
						\caption{Segnale flash telefono}
					\end{figure}
				\end{enumerate}
			\end{column}
		\end{columns}
		In esposizione a luce: disturbi periodici.
	\end{frame}
	
	
	
	% ---------- Linearità ----------
	\begin{frame}{Linearità $\Ipd$ vs $\Iled$ (LED $\to$ Fotodiodo)}
		
		\begin{columns}[T]
			
			% -------- COLONNA SINISTRA: testo + formule, avanzamento automatico --------
			\begin{column}{0.48\textwidth}
				\textbf{Procedura:}
				\begin{itemize}[<+->] % ogni item/formula avanza di uno
					\item \textbf{Setup:} Allineare il LED al PD; sweep di W1 come per le curve IV del LED.
					\item \textbf{Calcolo Correnti:}
					\[
					\Iled=\frac{V_{\mathrm{Ch1}}-V_{\mathrm{Ch2}}}{R_1}
					\]
					\item \[
					\Ipd=-\frac{V_{\text{out}}}{R_2}
					\]
					\item \textbf{Fit lineare:}\quad
					\[
					\Ipd = a\,\Iled + b
					\]
				\end{itemize}
			\end{column}
			
			% -------- COLONNA DESTRA: immagini che si sostituiscono --------
			\begin{column}{0.48\textwidth}
				
				% Aspetta che il testo a sinistra sia finito, poi inizia la giostra immagini
				\onslide<+->{
					\begin{overlayarea}{\columnwidth}{0.9\textheight}
						
						% Ogni \only<+> usa il PROSSIMO passo disponibile: niente numeri assoluti!
						\only<+>{
							\begin{figure}\centering
								\includegraphics[width=\linewidth]{Fit_LED_rosso.png}
								\caption{Fit per LED rosso}
							\end{figure}
						}
						
						\only<+>{
							\begin{figure}\centering
								\includegraphics[width=\linewidth]{Fit_LED_arancio.png}
								\caption{Fit per LED arancio}
							\end{figure}
						}
						
						\only<+>{
							\begin{figure}\centering
								\includegraphics[width=\linewidth]{Fit_LED_giallo.png}
								\caption{Fit per LED giallo}
							\end{figure}
						}
						
						\only<+>{
							\begin{figure}\centering
								\includegraphics[width=\linewidth]{Fit_LED_verde.png}
								\caption{Fit per LED verde}
							\end{figure}
						}
						
						\only<+>{
							\begin{figure}\centering
								\includegraphics[width=\linewidth]{Fit_LED_blu.png}
								\caption{Fit per LED blu}
							\end{figure}
						}
						
						\only<+>{
							\begin{figure}\centering
								\includegraphics[width=\linewidth]{Fit_LED_viola.png}
								\caption{Fit per LED viola}
							\end{figure}
						}
						
					\end{overlayarea}
				} % fine onsilde<+->
				
			\end{column}
		\end{columns}
	\end{frame}
	
	
	\begin{frame}{Linearità $\Ipd$ vs $\Iled$}
		
		\begin{columns}[T,totalwidth=\textwidth]
			% ====== COLONNA SINISTRA: testo e formule ======
			\begin{column}{0.48\textwidth}
				
				% --- 1) Testo con formule ---
				\textbf{Conversione fotometrica $\to$ radiometrica:}
				\[
				W_{PD}^{(\mathrm{ds})} = \frac{I_v\,A_{PD}}{683\,V(\lambda)\,d^2}
				\]
				\[
				W_{PD}^{(\mathrm{exp})} = \frac{I_{PD}}{\mathcal{R}(\lambda)}
				\]
				
				% --- 3) Interpretazione (mostrata dopo l'immagine) ---
				\onslide<3->{
					\vspace{0.5em}
					\small
					\textbf{Interpretazione:}
					\begin{itemize}
						\item Tutti i LED si trovano sotto la linea $y=x$ (solo una frazione del fascio raggiunge il PD).
						\item Il trend delle potenze è coerente con i datasheet.
						\item Differenze dovute a geometria e sensibilità spettrale del fotodiodo.
					\end{itemize}
				}
				
			\end{column}
			
			% ====== COLONNA DESTRA: immagine ======
			\begin{column}{0.5\textwidth}
				\onslide<2->{
					\begin{figure}
						\centering
						\includegraphics[width=\linewidth,keepaspectratio]{w.png}
						\caption{\scriptsize Potenza ottica $W$ in funzione di $I_{\!LED}$}
					\end{figure}
				}
			\end{column}
			
		\end{columns}
	\end{frame}
	
	
	
	% ---------- Conclusioni ----------
	\begin{frame}{Conclusioni}
		\begin{itemize}
			\item IV dei LED coerenti con Shockley; $\Vsoglia$ cresce all’aumentare di $1/\lambda$.
			\item Fit lineare $\Vsoglia$ vs $1/\lambda$ $\Rightarrow$ stima di $h$ compatibile con CODATA (entro $\sim10\%$).
			\item Transimpedenza: misura pulita di $\Ipd$; buona linearità $\Ipd \propto \Iled$ nel rosso.
		\end{itemize}
	\end{frame}
	
	% ---------- Backup: formule ----------
	\appendix
	\begin{frame}{(Backup) Formule utili}
		\begin{itemize}
			\item Shockley: $I=I_S\!\left(e^{V/\VT}-1\right)$,\quad $\VT=\dfrac{k_B T}{e}\approx\SI{25.85}{mV}$.
			\item Relazioni di calcolo: \ $\Iled=\dfrac{V_{\text{Ch1}}-V_{\text{Ch2}}}{R_1}$,\quad $\Ipd=-\dfrac{V_{\text{out}}}{R_2}$.
			\item Fit per $h$: \ $\Vsoglia = k + m(1/\lambda)$,\quad $h = m\,e/c$ \ (se $\lambda$ in nm: $h = m\,e\,10^{-9}/c$).
			\item Regime fotoconduttivo: $\displaystyle \Ipd=\eta(\nu)\frac{e}{h\nu}W$ \quad (responsività $R=\eta e/h\nu$).
		\end{itemize}
	\end{frame}
	
	\begin{frame}{(Backup) Spazi per figure e tabelle}
		
		\begin{columns}[T,totalwidth=\textwidth]
			% ====== COLONNA SINISTRA: Immagine ======
			\begin{column}{0.48\textwidth}
				\begin{figure}
					\centering
					\includegraphics[width=\linewidth]{response.png}
					\caption{Risposta spettrale del fotodiodo}
				\end{figure}
			\end{column}
			
			% ====== COLONNA DESTRA: Tabella ======
			\begin{column}{0.5\textwidth}
				\small
				\setlength{\tabcolsep}{6pt} % spazio tra colonne
				\renewcommand{\arraystretch}{1.2} % spazio verticale tra righe
				\begin{table}
					\centering
					\begin{tabular}{c S[scientific-notation = true, table-format=1.2e-3] S[scientific-notation = true, table-format=1.2e-3]}
						\toprule
						\textbf{LED (colore)} & 
						{$W_{20\,\mathrm{mA}}^{(\mathrm{meas})}\,[\mathrm{W}]$} & 
						{$W_{20\,\mathrm{mA}}^{(\mathrm{ds})}\,[\mathrm{W}]$} \\
						\midrule
						Arancio & 1.68e-5 & 2.86e-3 \\
						Rosso   & 5.37e-6 & 1.88e-3 \\
						Giallo  & 1.48e-5 & 1.22e-3 \\
						Verde   & 1.62e-4 & 3.44e-3 \\
						Blu     & 4.26e-4 & 2.52e-3 \\
						Viola   & 1.09e-3 & 7.24e-3 \\
						\bottomrule
					\end{tabular}
					\caption{\scriptsize Confronto tra potenze misurate e da datasheet a 20 mA.}
				\end{table}
			\end{column}
		\end{columns}
		
	\end{frame}
	
	\begin{frame}{Parte 2: Concetti teorici di base}
		\begin{columns}[T]
			\column{0.52\linewidth}
			
			\begin{itemize}
				\item Legge di Lambert-Beer: modello per quantificare il processo di assorbimento della radiazione elettromagnetica. Stima probabilità di assorbimento in termini di sezione d'urto di assorbimento e densità di centri assorbitori.
				\[
				I(x) = I_0 e^{- \alpha (\lambda) x}
				\]
				x coordinata all'interno del materiale
				
				\item Definisco poi $T \equiv \frac{I(d)}{I_0}$ trasmittanza e $A = - \log \left( \frac{I(d)}{I_0} \right)$ assorbanza, dove $d$ è lo spessore del materiale.
				
			\end{itemize}
			
			\column{0.44\linewidth}
			
			\begin{itemize}
				\item Nel caso dei liquidi $\alpha(\lambda) = c \epsilon(\lambda)$, dove $c$ è la concetrazione della sostanza di interesse nel liquido e $\epsilon(\lambda)$ è il coefficiente di assorbimento molare.
				
				\item In conclusione, nel caso in esame scrivo: $A = [c \epsilon(\lambda)] d$
			\end{itemize}
			
			
		\end{columns}
		
	\end{frame}
	
	\begin{frame}{Parte 2: Obiettivi e outline}
		\begin{block}{Passaggi principali}
			\begin{itemize}
				\item Preparazione campioni liquido di interesse a varie concentrazioni.
				\item Acquisizioni fotocorrenti del fotodiodo generate da luce emessa dai vari LED a disposizione che attraversa i vari campioni.
				\item Fit lineare per ogni lunghezza d'onda per stimare il relativo coefficiente di assorbimento.
				\item Spettro del coefficiente di assorbimento e comparazione con risultati presenti in letteratura.
			\end{itemize}
		\end{block}
	\end{frame}
	
	\begin{frame}{Parte 2: Preparazione campioni di liquido}
		
		Sono state utilizzate 5 cuvette, riempite ciascuna con una parte di Powerade ed una di acqua, di seguito le concentrazioni ottenute:
		
		\begin{table}[h]
			\centering
			\begin{tabular}{ccc}
				\toprule
				\textbf{Conc [V]} & \(\mathbf{V_W}\) & \(\mathbf{V_L}\) \\
				\midrule
				$0.0 \pm 0.1$ & $4.0 \pm 0.5$ & $0.0 \pm 0.5$ \\
				$0.3 \pm 0.1$ & $3.0 \pm 0.5$ & $1.0 \pm 0.5$ \\
				$0.50 \pm 0.09$ & $2.0 \pm 0.5$ & $2.0 \pm 0.5$ \\
				$0.8 \pm 0.1$ & $1.0 \pm 0.5$ & $3.0 \pm 0.5$ \\
				$1.0 \pm 0.1$ & $0.0 \pm 0.5$ & $4.0 \pm 0.5$ \\
				\bottomrule
			\end{tabular}
			\caption{Tabella dei valori delle concentrazioni di Powerade diluito in acqua utilizzate}
		\end{table}
		
	\end{frame}
	
	\begin{frame}{Parte 2: Acquisizione fotocorrenti}
		
		Descrizione presa dati:
		
		\begin{itemize}
			\item Per ogni LED si è selezionato un voltaggio al quale questo aveva un'emissione il più possibile brillante.
			\item Al dato voltaggio di alimentazione si è campionato più volte il potenziale del fotodiodo con sweepbias.ipynb, cambiando di volta in volta le cuvette, passando a campioni sempre più concentrati di powerade.
			\item Il potenziale è stato tradotto in corrente, proporzionale all'intensità luminosa, da questa si è calcolata l'assorbanza per ogni lunghezza d'onda studiata.
			L'assorbanza è stata calcolata misurando per ogni acquisizione anche la fotocorrente prodotta dalla luce dei LED in uscita dalla cuvetta piena d'acqua (intensità luminosa di riferimento)
			
			\item Si è esguito un fit lineare della concentrazione in funzione dell'assorbanza con il modello: 
			
			\[
			\frac{V_L}{V_W + V_L} =  - \log \frac{I(d)}{I_0} \cdot M + C
			\]
			
		\end{itemize}
		
	\end{frame}
	
	\begin{frame}{Parte 2. Fit e risultati ottenuti.}
		Sono stati eseguiti 6 fit, uno per ogni LED (rosso, arancione, giallo, verde, blu, e viola), di seguito si riportano i risultati del fit per il LED rosso a scopo di esempio:
		
		\vspace{0.5cm}
			
			\begin{table}[h]
				\centering
				\begin{tabular}{c c}
					\hline
					\textbf{Parametro} & \textbf{Valore} \\
					\hline
					$\chi^{2}_{\mathrm{rid}}$ & $0.7 \pm 0.8$ \\
					M & $11 \pm 2$ \\
					Q & $-0.02 \pm 0.09$ \\
					\hline
				\end{tabular}
			\end{table}
		
		\vspace{0.5cm}
		
		Per tutti gli altri fit sono stati ottenuti $\chi^2_{rid}$ simili, cioè entro una sigma da 1 e offset tutti confrontabili con zero, a supporto della validità dei risultati ottenuti.
		
	\end{frame}
	
	\begin{frame}{Parte 2. Fit e risultati ottenuti.}
		
		\begin{figure}
			\centering
			\includegraphics[width=0.9\textwidth, height=0.75\textheight, keepaspectratio]{plot_fit_conc_pres_vs_A_rosso.png}
			\caption{Grafico di fit per LED rosso: concentrazione campioni vs assorbanza.}
		\end{figure}
		
	\end{frame}

	\begin{frame}{Parte 2. Spettro del coefficiente di assorbimento.}
		
		Da ogni fit concentrazione vs assorbanza dei LED è stato ricavato il coefficiente di assorbimento del liquido, alla lunghezza del LED utilizzato; i valori ottenuti sono riportati nella seguente tabella:
		
		\vspace{0.5cm}
		
		\begin{table}[h]
			\centering
			\begin{tabular}{c c c}
				\hline
				\textbf{Colore} & $\boldsymbol{\lambda\,[\mathrm{nm}]}$ & $\boldsymbol{\alpha\,[\mathrm{m^{-1}}]}$ \\
				\hline
				Rosso      & $625 \pm 7$        & $7 \pm 1$ \\
				Arancione  & $604 \pm 8$        & $46 \pm 8$ \\
				Giallo     & $588 \pm 7$        & $6 \pm 1$ \\
				Verde      & $532 \pm 13$       & $31 \pm 5$ \\
				Blu        & $469 \pm 7$        & $(9 \pm 2)\cdot 10$ \\
				Viola      & $401 \pm 5$        & $(7 \pm 1)\cdot 10$ \\
				\hline
			\end{tabular}
			\caption{Valori di lunghezza d’onda e coefficienti di assorbimento.}
		\end{table}
		
		\vspace{0.5cm}
		
	\end{frame}


\begin{frame}{Parte 2. Spettro del coefficiente di assorbimento.}
	
	\begin{figure}
		\centering
		\includegraphics[width=0.9\textwidth, height=0.75\textheight, keepaspectratio]{plot_coeffasb_vs_lambda1.png}
		\caption{Coefficiente di assorbimento Powerade in funzione della lunghezza d'onda dei LED utilizzati.}
	\end{figure}
	
\end{frame}


	
\end{document}
