% !TEX program = pdflatex
% =============================================================================
% S09 - Registri a scorrimento e generatore pseudocasuale
% =============================================================================

\documentclass[aspectratio=169,10pt]{beamer}

% --- Tema stabile -------------------------------------------------------------
\usetheme{Madrid}
\setbeamertemplate{footline}[frame number]
\setbeamertemplate{navigation symbols}{}

% --- Pacchetti ----------------------------------------------------------------
\usepackage[utf8]{inputenc}
\usepackage[T1]{fontenc}
\usepackage[italian]{babel}
\usepackage{lmodern}
\usepackage{microtype}
\usepackage{siunitx}
\sisetup{per-mode=symbol,detect-all}
\usepackage{graphicx}
\usepackage{xspace}
\usepackage{booktabs}
\usepackage{tikz}

% --- Palette ------------------------------------------------------------------
\definecolor{tdBlue}{RGB}{13,59,102}
\definecolor{tdCyan}{RGB}{0,167,196}
\definecolor{tdGreen}{RGB}{46,212,119}
\definecolor{tdDark}{RGB}{11,12,16}

\setbeamercolor{normal text}{fg=tdDark,bg=white}
\setbeamercolor{structure}{fg=tdBlue}
\setbeamercolor{frametitle}{fg=white,bg=tdBlue}
\setbeamercolor{block title}{fg=white,bg=tdBlue}
\setbeamercolor{block body}{bg=tdBlue!5}

% --- Banner di copertina ------------------------------------------------------
\newcommand{\TitleBanner}{%
  \begin{tikzpicture}[remember picture,overlay]
    \node[anchor=north west, inner sep=0pt] at (current page.north west)
      {\includegraphics[width=\paperwidth]{copertina.png}};
  \end{tikzpicture}
  \vspace{0.28\paperheight}
}
\addtobeamertemplate{title page}{\TitleBanner}{}

% --- Icone testuali -----------------------------------------------------------
\newcommand{\NAND}{\textsc{NAND}\xspace}
\newcommand{\NOT}{\textsc{NOT}\xspace}
\newcommand{\XOR}{\textsc{XOR}\xspace}
\newcommand{\FF}{\textsc{FF}\xspace}

% --- ROADMAP ---------------------------------------------------------------
\newcommand{\RoadBoxActive}[2]{%
  \setbeamercolor{roadbox}{bg=tdBlue, fg=white}
  \begin{beamercolorbox}[wd=\dimexpr.28\linewidth\relax,ht=1.15cm,dp=3pt,center,rounded=true]{roadbox}
    \large \hyperlink{#1}{#2}
  \end{beamercolorbox}
}
\newcommand{\RoadBoxInactive}[2]{%
  \setbeamercolor{roadbox}{bg=tdBlue!10, fg=tdDark}
  \begin{beamercolorbox}[wd=\dimexpr.28\linewidth\relax,ht=1.15cm,dp=3pt,center,rounded=true]{roadbox}
    \large \hyperlink{#1}{#2}
  \end{beamercolorbox}
}

% Selettore
\newcommand{\RoadPick}[3]{%
  \ifnum\value{section}=#1
    \RoadBoxActive{#2}{#3}%
  \else
    \RoadBoxInactive{#2}{#3}%
  \fi
}

% Roadmap frame
\renewcommand{\RoadmapFrame}{%
  \begin{frame}{Roadmap}
    \centering
    \vfill
    \begin{minipage}{0.8\linewidth}
      \makebox[\linewidth]{%
        \RoadPick{1}{sec:intro}{Obiettivi}\hfill
        \RoadPick{2}{sec:div}{Divisori D-FF}%
      }\\[12mm]
      \makebox[\linewidth]{%
        \RoadPick{3}{sec:shift}{Registri MC14557}\hfill
        \RoadPick{4}{sec:lfsr}{Generatore pseudocasuale}%
      }\\[12mm]
      \makebox[\linewidth]{%
        \RoadPick{5}{sec:issues}{Problemi \& soluzioni}\hfill
        \RoadPick{6}{sec:end}{Conclusioni}%
      }
    \end{minipage}
    \vfill
  \end{frame}
}

\AtBeginSection[]{\RoadmapFrame}

% --- Metadati -----------------------------------------------------------------
\title{S09 -- Registri a scorrimento e generatore pseudocasuale}
\subtitle{Dai divisori ai LFSR: MC14557, CD4013 e ciclo di feedback}
\author{Nome Cognome \\ \small T10}
\institute{Tecnologie Digitali}
\date{\today}

% ============================ DOCUMENTO =======================================
\begin{document}

% TITOLO
\begin{frame}
  \titlepage
\end{frame}

% ============================ SEZIONE 1 =======================================
\section{Obiettivi}
\begin{frame}{Obiettivi della sessione}\hypertarget{sec:intro}{}
  \begin{block}{Cosa abbiamo costruito e cosa abbiamo analizzato}
    \begin{itemize}
      \item Divisori di frequenza con \FF{} D (CD4013).
      \item Registro a scorrimento programmabile MC14557.
      \item Misura del \textit{ritardo di propagazione} per vari valori di \(n\).
      \item Generatore di bit pseudocasuali tramite XOR e feedback.
      \item Analisi della periodicità e confronto con la teoria.
    \end{itemize}
  \end{block}
\end{frame}

% ============================ SEZIONE 2 =======================================
% ==================== DIVISORE PER 2 =========================================
\section{Divisori D-FF}
\begin{frame}{Divisore di frequenza per 2}\hypertarget{sec:div}{}
  \vfill
  \begin{columns}[c,onlytextwidth]

    % --------- Colonna sinistra: testo + schema (con comparsa) ---------------
    \column{0.52\textwidth}
    \begin{itemize}[<+->]
      \item Collegamento: \(\overline{Q} \to D\), clock su fronte di salita.
      \item In uscita si ottiene \(f_{\text{out}} = f_{\text{in}}/2\).
      \item È il blocco base per contatori, divisori e registri a scorrimento.
    \end{itemize}

    \vspace{3mm}

    \onslide<2->{%
      \begin{figure}
        \centering
        % schema del divisore per 2 (flip–flop singolo con feedback su D)
        \includegraphics[width=\linewidth]{schema_divisore2.png}
        \caption{\small\textit{Schema del divisore di frequenza per 2 con \FF{} D.}}
      \end{figure}
    }

    % --------- Colonna destra: forme d’onda misurate -------------------------
    \column{0.48\textwidth}
    \onslide<3->{%
      \begin{figure}
        \centering
        \includegraphics[width=\linewidth]{Divisore_fmezzi.pdf}
        \caption{\small\textit{Forme d’onda sperimentali: clock in ingresso (Ch1) e uscita divisa per 2 (Ch2).}}
      \end{figure}
    }

  \end{columns}
  \vfill
\end{frame}

% ==================== DIVISORE PER 3 =========================================
\begin{frame}{Divisore di frequenza per 3}
  \vfill
  \begin{columns}[c,onlytextwidth]

    % --------- Colonna sinistra: testo + schema + tabella --------------------
    \column{0.52\textwidth}

    % Testo che compare per primo
    \begin{itemize}[<+->]
      \item Due \FF{} D in cascata con rete di feedback logica.
      \item Condizione di feedback: \(IN = \overline{Q_1^n}\cdot Q_2^n\).
      \item In regime stazionario si ottiene \(f_{\text{out}} = f_{\text{in}}/3\).
    \end{itemize}

    \vspace{2mm}

    % Schema circuitale + tabella di verità (secondo step)
    \onslide<2->{%

      \vspace{1mm}

      \centering
      \small
      \begin{tabular}{cccc}
        \toprule
        Stato \(n\) &
        \(IN = \overline{Q_1^n}\cdot Q_2^n\) &
        Stato \(n+1\) &
        \(OUT = Q_2^{n+1}\) \\
        \midrule
        00 & 1 & 10 & 0 \\
        10 & 1 & 11 & 1 \\
        01 & 1 & 10 & 0 \\
        11 & 0 & 01 & 1 \\
        \bottomrule
      \end{tabular}
    }

    % --------- Colonna destra: forme d’onda misurate -------------------------
    \column{0.48\textwidth}
    \onslide<3->{%
    \begin{figure}
        \centering
        \includegraphics[width=0.4\linewidth]{schema_divisore_freq_task3.png}
        
      \end{figure}
      \caption{\small\textit{Schema del divisore di frequenza per 3 con due \FF{} D e porta \NAND/\AND\ di feedback.}}
      \begin{figure}
        \centering
        \includegraphics[width=0.9\linewidth]{Divisore_fterzi.pdf}
        
      \end{figure}
      \caption{\small\textit{Forme d’onda sperimentali}}
    }

  \end{columns}
  \vfill
\end{frame}


% ============================ SEZIONE 3 =======================================
\section{Registro MC14557}

% --- SLIDE 1 MODIFICATA ---
\begin{frame}{Registro lineare a lunghezza variabile}\hypertarget{sec:shift}{}
\begin{columns}[c] % [c] per centrare verticalmente il contenuto
 \centering
 \column{0.58\textwidth}
    % [<+->] fa apparire gli elementi uno per uno
 \begin{itemize}[<+->]
 \item Il chip contiene blocchi da 1, 2, 4, 8, 16 e 32 bit.
 \item Gli ingressi \(L_i\) selezionano quali blocchi attivare.
 \item Lunghezza: \(n = 1 + L_1 + 2L_2 + \dots + 32L_{32}\).
 \item Ingressi cruciali: A/B, CE, RESET.
 \end{itemize}

 \column{0.42\textwidth}
    % L'immagine appare dopo i 4 punti elenco (quindi allo step 5)
    \onslide<5->{%
      \begin{figure}
        \centering
        \includegraphics[width=\linewidth]{piedinatura_MC14557.png}
        
      \end{figure}
      \caption{\small Piedinatura e schema a blocchi del MC14557.}
    }
 \end{columns}
\end{frame}

% --- SLIDE 2 MODIFICATA ---
\begin{frame}{Ritardo di propagazione}
  \begin{columns}[c]
    % --- Colonna Sinistra: Testo ---
    \column{0.55\textwidth}
    \centering
    % [<+->] fa apparire i 3 punti elenco in sequenza
 \begin{itemize}[<+->]
 \item Misura: differenza temporale tra ingresso \(A\) e uscita \(Q\).
 \item Ritardo oscillante con \(n\).
 \item Necessario sincronizzare W1 e W2.
 \end{itemize}

    % --- Colonna Destra: Immagini in stack ---
    \column{0.45\textwidth}
    % Prima immagine: appare allo step 4 (dopo i 3 punti elenco)
    \onslide<4->{%
      \begin{figure}
        \centering
        % Imposto un'altezza massima per farle stare entrambe
        \includegraphics[width=\linewidth, height=0.35\textheight, keepaspectratio]{reg_scorr_32bit.png}
        
      \end{figure}
      \centering
      \caption{\small Forme d'onda 32 bit (sopra), 64 bit (sotto)}
    }
    \vspace{0.2em} % Un po' di spazio tra le immagini
    % Seconda immagine: appare allo step 5
    \onslide<5->{%
      \begin{figure}
        \centering
        \includegraphics[width=\linewidth, height=0.35\textheight, keepaspectratio]{reg_scorr_64bit.png}
      \end{figure}
            

    }
  \end{columns}
\end{frame}

% ============================ SEZIONE 4 =======================================
\section{Generatore pseudocasuale}
\begin{frame}{Architettura del generatore}\hypertarget{sec:lfsr}{}
  \begin{itemize}
    \item Feedback: XOR tra uscita del MC14557 e \FF{} CD4013.
    \item Innesco: caricare tutti 1 con ingresso A → quindi selezionare B.
  \end{itemize}
  \centering
  \includegraphics[width=0.4\linewidth, angle=90]{Circuito_realizzato_task8_conCavetto.jpg}
\end{frame}

\begin{frame}{Analisi della periodicità}
  \begin{columns}[c]
    % --- Colonna Sinistra: Testo e Tabella ---
    \column{0.58\textwidth}
    
    % Testo a comparsa
    \begin{itemize}[<+->]
      \item Dai calcoli emerge:
      \begin{itemize}
        \item \(n=1,2,3,5\): periodo max \(2^{m}-1\).
        \item \(n=4\): compaiono sottocicli, periodo ridotto ($L=21$).
      \end{itemize}
      \item Confronto tra teoria e misure:
    \end{itemize}

    % Tabella a comparsa (appare dopo il testo)
    \onslide<3->{
      \vspace{0.5em}
      \centering
      % \resizebox adatta la tabella alla larghezza della colonna
      \resizebox{\linewidth}{!}{%
        \begin{tabular}{c c l l c}
          \toprule
          \textbf{n} & \textbf{m=n+1} & \textbf{Stato iniz.} & \textbf{Sequenza teorica (da Q)} & \textbf{\(L_{teo}\)} \\
          \midrule
          1 & 2 & 11 & 1 1 0 & 3 \\
          2 & 3 & 111 & 1 1 0 1 1 1 1 & 7 \\
          3 & 4 & 1111 & 1 1 1 1 0 0 0 1 0 0 1 1 0 1 0 & 15 \\
          4 & 5 & 11111 & 1 1 1 1 1 0 0 0 0 1 0 0 0 1 1 0 0 1 0 1 0 & 21 \\
          5 & 6 & 111111 & 111111000001\dots (seq. max) & 63 \\
          \bottomrule
        \end{tabular}%
      }

    }

    % --- Colonna Destra: Confronto Grafico ---
    \column{0.42\textwidth}
    \onslide<4->{
      \begin{figure}
        \centering
        % Simulazione
        \includegraphics[width=\linewidth, height=0.35\textheight, keepaspectratio]{download.png}
        
        \vspace{0.5em} % Spazio tra le due immagini
        
        % Acquisizione reale
        \includegraphics[width=\linewidth, height=0.35\textheight, keepaspectratio]{task8_4bit_wire.png}
      \end{figure}
              \caption{\small Simulazione VS Acquisizione per ($n=4$).}

    }
  \end{columns}
\end{frame}

\begin{frame}{Estrazione sequenza di bit dal segnale analogico}

  % =================== COMPARSA 1: BLOCCO SPIEGAZIONE ======================
  \onslide<1->{
  \begin{block}{Come MATLAB ricava la sequenza di bit}
    \begin{itemize}
      \item Si identificano i fronti di salita del clock tramite soglia
            \(V_{\mathrm{thr,clk}} = 2.5\,\text{V}\).
      \item Si calcola il numero medio di campioni per periodo del clock 
            direttamente dai dati sperimentali.
      \item Si campiona l’uscita \(V_{\mathrm{out}}\) a metà di ogni periodo 
            \(\Rightarrow\) valore stabile \(\rightarrow\) bit \(0/1\).
      \item Il periodo \(L\) si determina trovando quando la sequenza si ripete.
    \end{itemize}
  \end{block}
  }

  \vspace{3mm}

  % =================== COMPARSA 2: IMMAGINI ======================
  \onslide<2->{
  \begin{columns}[T,onlytextwidth]

    % COLONNA SINISTRA: Clock
    \column{0.48\textwidth}
      \centering
      \includegraphics[width=\linewidth]{clock_campionato.png}
      \caption{\small Clock e fronti di salita (\(n=4\)).}

    % COLONNA DESTRA: Uscita campionata
    \column{0.48\textwidth}
      \centering
      \includegraphics[width=\linewidth]{ch2_campionato.png}
      \caption{\small Uscita campionata e bit estratti (\(n=4\)).}

  \end{columns}
  }

\end{frame}

\begin{frame}{Confronto uscita: MATLAB vs misura}

  \centering

  \begin{columns}[c,onlytextwidth]
    % ===================== COLONNA SINISTRA: MATLAB ======================
    \column{0.48\textwidth}
      \centering
      \only<1>{
        \includegraphics[width=\linewidth]{1bit_campionato.png}\\[1mm]
        {\small MATLAB, registro a \(n=1\) bit}
      }
      \only<2>{
        \includegraphics[width=\linewidth]{2bit_campionato.png}\\[1mm]
        {\small MATLAB, registro a \(n=2\) bit}
      }
      \only<3>{
        \includegraphics[width=\linewidth]{3bit_campionato.png}\\[1mm]
        {\small MATLAB, registro a \(n=3\) bit}
      }
      \only<4>{
        \includegraphics[width=\linewidth]{ch2_campionato.png}\\[1mm]
        {\small MATLAB, registro a \(n=4\) bit}
      }
      \only<5>{
        \includegraphics[width=\linewidth]{5bit_campionato.png}\\[1mm]
        {\small MATLAB, registro a \(n=5\) bit}
      }

    % ===================== COLONNA DESTRA: SPERIMENTALE ==================
    \column{0.48\textwidth}
      \centering
      \only<1>{
        \includegraphics[width=\linewidth]{task8_1bit_wire.png}\\[1mm]
        {\small Misura sperimentale, \(n=1\)}
      }
      \only<2>{
        \includegraphics[width=\linewidth]{task8_2bit_wire.png}\\[1mm]
        {\small Misura sperimentale, \(n=2\)}
      }
      \only<3>{
        \includegraphics[width=\linewidth]{task8_3bit_wire.png}\\[1mm]
        {\small Misura sperimentale, \(n=3\)}
      }
      \only<4>{
        \includegraphics[width=\linewidth]{task8_4bit_wire_nuove.png}\\[1mm]
        {\small Misura sperimentale, \(n=4\)}
      }
      \only<5>{
        \includegraphics[width=\linewidth]{task8_5bit_wire.png}\\[1mm]
        {\small Misura sperimentale, \(n=5\)}
      }
  \end{columns}

  \vspace{2mm}
  \onslide<1-5>{
    
  }

\end{frame}

\begin{frame}{Periodo massimo di un LFSR}

  \begin{block}{Obiettivo}
    Variando il numero di bit \(n\) del registro a scorrimento si vuole
    determinare il \textbf{massimo periodo di ripetizione} osservabile
    con il circuito montato, usando tre punti di vista:
    \begin{itemize}[<+->]
      \item \textbf{Teorico}: previsione del periodo di un LFSR a \(n\) bit
            con polinomio caratteristico primitivo (\(L = 2^n-1\)).
      \item \textbf{Numerico}: simulazione dell’LFSR e calcolo esplicito
            del periodo della sequenza.
      \item \textbf{Sperimentale}: misura dell’uscita del registro e
            ricostruzione della stringa di bit.
    \end{itemize}
  \end{block}

  \vspace{3mm}

  \uncover<4->{
    \begin{block}{Strategia}
      Si utilizza il chip MC14557 configurato con un certo numero di bit
      attivi: dapprima \(n=7\) (registro a 8 bit), poi \(n=6\),
      confrontando il periodo atteso con quello misurato.
    \end{block}
  }

\end{frame}
\begin{frame}{Scelta dei parametri di misura}

  \begin{block}{Vincoli sperimentali}
    \begin{itemize}[<+->]
      \item Più bit \(\Rightarrow\) periodo più lungo, ma anche
            \textbf{tempo di misura} più lungo.
      \item AD2: massimo \(\SI{1e8}{Sa/s}\), ma con ADC a 14 bit il numero
            di punti utili è limitato (\(\sim 1.6\times10^4\)).
      \item Si vuole mantenere:
        \begin{itemize}
          \item numero di bit \(n\) il più alto possibile;
          \item frequenza di clock non troppo elevata, per avere
                \(\ge 4\text{--}5\) campioni per livello logico;
          \item numero di punti campionati il più alto possibile.
        \end{itemize}
    \end{itemize}
  \end{block}

  \vspace{2mm}

  \begin{block}{Scelte finali dei parametri} \small
    \begin{itemize}[<+->]
      \item Frequenza di campionamento:
            \(f_{\mathrm{sampling}} = \SI{1e5}{Sa/s}\), con \(N = 10000\) punti
            \(\Rightarrow\) finestra temporale \(\approx \SI{0.1}{s}\).
      \item Generatore W2 lento, \(f_{W2} = \SI{5}{Hz}\),
            in modo che la semionda bassa duri \(\sim\SI{0.1}{s}\)
            \(\Rightarrow\) si misura un singolo periodo durante la semionda bassa.
      \item \textbf{Configurazione iniziale}: MC14557 a \(7\) bit
            \(\Rightarrow\) LFSR a \(8\) bit, periodo atteso
            \(L_{\mathrm{teo}} = 2^8-1 = 255\) stati
            (in assenza di sottocicli).
      \item Frequenza di clock scelta:
            \(f_{\mathrm{clock}} \approx \SI{2550}{Hz}\),
            circa 5 volte inferiore a \(f_{\mathrm{sampling}}\).
    \end{itemize}
  \end{block}

\end{frame}
\begin{frame}{Configurazione MC14557 a 7 bit (LFSR a 8 bit)}

  \begin{columns}[c,onlytextwidth]
    \column{0.52\textwidth}
      \begin{itemize}[<+->]
        \item Teoricamente, con polinomio primitivo ci si aspetta
              un periodo \(L_{\mathrm{teo}} = 2^8 - 1 = 255\) stati.
        \item In misura si osserva però una sequenza che si ripete
              \(\sim 4\) volte nella finestra di \(\SI{0.1}{s}\).
        \item Stimando il numero di bit distinti nel singolo ciclo
              si trova un \textbf{sottociclo} di circa \(63\) stati,
              molto inferiore a \(255\).
        \item Conclusione: la combinazione di tap usata sul MC14557
              non corrisponde a un polinomio primitivo a 8 bit,
              quindi l’LFSR non esplora tutti gli stati possibili.
      \end{itemize}

    \column{0.48\textwidth}
      \centering
      \includegraphics[width=\linewidth]{task10_misura1.png}
      \caption{\small Acquisizione con MC14557 configurato a 7 bit:
               il ciclo si ripete più volte nel tempo di osservazione,
               evidenziando un sottociclo.}
  \end{columns}

\end{frame}
\begin{frame}{Configurazione MC14557 a 6 bit (LFSR a 7 bit)}

  \begin{block}{Nuova configurazione}
    \begin{itemize}[<+->]
      \item Per evitare i sottocicli si riduce il numero di bit:
            MC14557 configurato a \(6\) bit \(\Rightarrow\) LFSR a \(7\) bit.
      \item Teoricamente ci si aspetta un periodo
            \(L_{\mathrm{teo}} = 2^7 - 1 = 127\) stati.
      \item Si mantiene la stessa strategia di misura:
            \(f_{\mathrm{sampling}} = \SI{1e5}{Sa/s}\),
            \(N = 10000\) punti,
            \(f_{\mathrm{clock}} \approx \SI{1270}{Hz}\).
    \end{itemize}
  \end{block}

  \vspace{2mm}

  \begin{columns}[c,onlytextwidth]
    \column{0.5\textwidth}
      \centering
      \includegraphics[width=\linewidth]{task10_misura2.png}
      \caption{\small Acquisizione di un periodo (circa) del ciclo a 7 bit.}

    \column{0.5\textwidth}
      \centering
      \includegraphics[width=\linewidth]{task10_misura2_1.png}
      \caption{\small Acquisizione con clock doppio: si osservano due periodi completi.}
  \end{columns}

\end{frame}
\begin{frame}{Confronto teorico, numerico e sperimentale (n = 6)}

  \begin{block}{Periodo atteso e calcolo numerico}
    \begin{itemize}[<+->]
      \item \textbf{Teorico}: per l’LFSR a \(7\) bit con polinomio scelto
            ci si aspetta una stringa binaria di
            \(L_{\mathrm{teo}} = 127\) bit che si ripete ciclicamente.
      \item \textbf{Numerico}: ripetendo le operazioni dell’Homework 1
            si ottiene una stringa binaria teorica di lunghezza 127,
            che definisce il contenuto del registro a ogni colpo di clock.
      \item \textbf{Sperimentale}: applicando il codice MATLAB del Task 9
            ai dati misurati si ricostruisce la stringa binaria
            sperimentale.
    \end{itemize}
  \end{block}

  \vspace{2mm}

  \begin{columns}[c,onlytextwidth]
    \column{0.52\textwidth}
      \small
      \uncover<4->{
      \begin{itemize}
        \item Confrontando le due stringhe si vede che:
          \begin{itemize}
            \item coincidono bit-a-bit per i primi \(127\) bit;
            \item dopo il 127-esimo bit la sequenza sperimentale
                  ricomincia dall’inizio, come previsto.
          \end{itemize}
        \item Non si osservano sottocicli: il periodo sperimentale
              coincide con quello teorico.
      \end{itemize}
      }

    \column{0.48\textwidth}
      \centering
      \uncover<3->{
        \includegraphics[width=\linewidth]{6bit_campionato.png}
        \caption{\small Uscita campionata e bit estratti (\(n = 6\)):
                 la stringa ricostruita ha periodo \(L = 127\).}
      }
  \end{columns}

\end{frame}




% ============================ SEZIONE 5 =======================================
\section{Problemi riscontrati e soluzioni}
\begin{frame}{Criticità tipiche}\hypertarget{sec:issues}{}
  \begin{itemize}
    \item \textbf{Pin flottanti}: CAUSA \(\Rightarrow\) instabilità logica. SOLUZIONE: fissare CE, RESET, A/B.
    \item \textbf{Carico del LED}: abbassa Q → usare resistenza elevata.
  \end{itemize}
\end{frame}

% ============================ SEZIONE 6 =======================================
\section{Conclusioni}
\begin{frame}\hypertarget{sec:end}{}
  \begin{center}
    \Huge \textcolor{tdBlue}{\textbf{Grazie per l'attenzione!}}\\[1em]
    \Large Domande?
  \end{center}
\end{frame}




\end{document}
